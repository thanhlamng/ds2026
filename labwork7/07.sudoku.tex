\documentclass{article}
\usepackage[utf8]{inputenc}
\usepackage{geometry}
\usepackage{listings}
\usepackage{amsmath}
\usepackage{float}

\geometry{a4paper, margin=1in}

\title{Practical Work 7: Sudoku Game on Google App Engine (GAE)}
\author{Student Name: Bui Duc Minh}
\date{\today}

\lstset{
    language=bash, % Default language for deployment steps
    basicstyle=\footnotesize\ttfamily,
    numbers=left,
    numberstyle=\tiny,
    frame=single,
    showstringspaces=false,
    captionpos=b,
    breaklines=true,
    breakatwhitespace=true,
}

\begin{document}
\maketitle

\section{Application Structure and Deployment}
The Sudoku game was developed as a basic web application adhering to the Platform as a Service (PaaS) model provided by Google App Engine (GAE). The application utilizes Python and the Flask microframework to run within GAE's sandboxed environment, leveraging automatic scaling.

\subsection{Deployment Steps and Configuration}

The following steps were executed to set up and deploy the application to GAE:

\subsubsection{Step 1: Project Initialization}
\begin{lstlisting}[caption=Project Initialization Steps, language=bash]
# 1. Access the Google Cloud Console and create a new project.
# 2. Enable the App Engine API for the created project.
# 3. Initialize the Google Cloud SDK locally:
gcloud init
\end{lstlisting}

\subsubsection{Step 2: Configuration (\texttt{app.yaml})}
The configuration file defines the environment and resource allocation. This utilizes Python 3.9 and Gunicorn for the web server processes.

\begin{lstlisting}[caption=GAE Configuration (\texttt{app.yaml}), language=yaml]
runtime: python39
entrypoint: gunicorn -w 4 main:app

handlers:
- url: /.*
  script: auto
\end{lstlisting}

\subsubsection{Step 3: Core Logic and Implementation Snippets}
The application logic, implemented in \texttt{main.py}, must handle three core tasks: board generation, cell masking, and submission validation. This is the implementation structure.

\begin{lstlisting}[caption=Core Logic Snippet (\texttt{main.py}), language=Python]
from flask import Flask, request, render_template_string

app = Flask(__name__)

# --- Core Sudoku Logic (Implementation) ---

def generate_valid_board():
    # Logic to generate a full, valid 9x9 board (e.g., using backtracking).
    return board

def create_puzzle(board):
    # Logic to mask (hide) cells to create the puzzle for the user.
    return puzzle_board

def validate_solution(user_solution):
    # Logic to check if the user's input adheres to Sudoku rules.
    return is_valid

@app.route('/', methods=['GET', 'POST'])
def game_handler():
    # 1. On GET: Generate new puzzle and render HTML.
    # 2. On POST: Receive user input and call validate_solution().
    return render_template_string(...)
\end{lstlisting}

\subsubsection{Step 4: Deployment}
The \texttt{gcloud app deploy} command finalizes the process, uploading the code and initiating containerization on Google's infrastructure.

\begin{lstlisting}[caption=Deployment Command, language=bash]
gcloud app deploy
\end{lstlisting}

\end{document}
